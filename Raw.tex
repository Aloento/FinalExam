\documentclass{article}

\usepackage[a4paper, margin=2cm]{geometry}


\begin{document}


\section{Algorithms and data structures}

The concept of algorithms, the description of an algorithm. Turing machines, the halting problem. Efficiency of algorithms, basics of complexity theory. Approaches to algorithm construction. Linear data structures (lists, collections, FIFO and LIFO stacks, binary heap) and their operations (sorting). Search in hash tables. Search based on binary trees. Search, insertion, deletion, balancing. Applications of B-trees in large databases.


\subsection{Linear data structures (lists, collections, FIFO and LIFO stacks, binary heap)}


\subsection{Search, insertion, deletion, balancing}

In this response, we'll discuss the search, insertion, deletion, and balancing operations for two common self-balancing binary search trees (BSTs): AVL trees and red-black trees.

**AVL Tree:**

1. **Search**: Searching in an AVL tree is the same as searching in a regular BST. The time complexity of searching in an AVL tree is O(log n) because it is always balanced.

2. **Insertion**: To insert a new element in an AVL tree, follow these steps:

   - Perform a regular BST insertion.
   - Update the height of the affected nodes.
   - Check for any imbalance in the tree (a balance factor of -2, 0, or 2). If any imbalance is found, perform rotations to rebalance the tree.

   The time complexity of insertion in an AVL tree is O(log n).

3. **Deletion**: To delete an element from an AVL tree, follow these steps:

   - Perform a regular BST deletion.
   - Update the height of the affected nodes.
   - Check for any imbalance in the tree (a balance factor of -2, 0, or 2). If any imbalance is found, perform rotations to rebalance the tree.

   The time complexity of deletion in an AVL tree is O(log n).

4. **Balancing**: AVL trees use rotations to maintain balance after insertion or deletion operations. There are four types of rotations: single right rotation, single left rotation, right-left rotation (a combination of right rotation followed by left rotation), and left-right rotation (a combination of left rotation followed by right rotation). The appropriate rotation type depends on the specific imbalance in the tree.

**Red-Black Tree:**

1. **Search**: Searching in a red-black tree is the same as searching in a regular BST. The time complexity of searching in a red-black tree is O(log n) because it is always balanced.

2. **Insertion**: To insert a new element in a red-black tree, follow these steps:

   - Perform a regular BST insertion, with the new node colored red.
   - Fix any red-black violations in the tree caused by the insertion, using rotations and recoloring.

   The time complexity of insertion in a red-black tree is O(log n).

3. **Deletion**: To delete an element from a red-black tree, follow these steps:

   - Perform a regular BST deletion.
   - Fix any red-black violations in the tree caused by the deletion, using rotations and recoloring.

   The time complexity of deletion in a red-black tree is O(log n).

4. **Balancing**: Red-black trees use rotations and recoloring to maintain the red-black properties after insertion or deletion operations. The specific steps to fix red-black violations depend on the relationship between the affected nodes (parent, sibling, and uncle) and their colors.

In summary, search, insertion, and deletion operations in both AVL trees and red-black trees have a time complexity of O(log n) due to their balanced nature. AVL trees are more strictly balanced than red-black trees and may require more rotations, making them slightly slower for insertion and deletion operations. However, AVL trees can have faster search times in practice. On the other hand, red-black trees are more efficient for insertion and deletion operations, making them a better choice for certain applications, such as language libraries and databases.


\subsection{balancing}

Balancing is the process of adjusting a binary search tree (BST) to maintain its height as minimal as possible to ensure optimal performance for search, insertion, and deletion operations. In a balanced tree, the difference in heights between the left and right subtrees of any node is small, typically 1 or less. Balancing is crucial in self-balancing BSTs, such as AVL trees and red-black trees, to ensure O(log n) time complexity for their operations.

Here's an overview of balancing in AVL trees and red-black trees:

**AVL Trees:**

AVL trees use rotations to maintain balance. The balance factor of a node is the difference between the heights of its left and right subtrees. In an AVL tree, the balance factor of every node must be -1, 0, or 1. If an insertion or deletion operation results in a balance factor outside this range, rotations are performed to restore balance. There are four types of rotations in AVL trees:

1. **Single Right Rotation (LL rotation)**: Occurs when the left subtree of the left child of a node is taller.
2. **Single Left Rotation (RR rotation)**: Occurs when the right subtree of the right child of a node is taller.
3. **Right-Left Rotation (RL rotation)**: Occurs when the right subtree of the left child of a node is taller. This rotation is a combination of a right rotation followed by a left rotation.
4. **Left-Right Rotation (LR rotation)**: Occurs when the left subtree of the right child of a node is taller. This rotation is a combination of a left rotation followed by a right rotation.

**Red-Black Trees:**

Red-black trees use both rotations and recoloring to maintain the red-black properties, which ensure balance. After an insertion or deletion operation, red-black tree balancing involves fixing any violations of the red-black properties:

1. Every node is either red or black.
2. The root node is black.
3. Every leaf node (null) is black.
4. If a node is red, both its children are black.
5. Every path from a node to its descendant leaf nodes contains the same number of black nodes.

To fix these violations, the tree may perform single or double rotations, as well as recoloring of nodes. The specific steps to fix violations depend on the relationship between the affected nodes (parent, sibling, and uncle) and their colors.

In summary, balancing is a crucial aspect of self-balancing binary search trees like AVL trees and red-black trees. It ensures that the tree remains balanced, providing optimal performance for search, insertion, and deletion operations. Balancing is achieved through rotations and, in the case of red-black trees, recoloring of nodes.


\subsection{Applications of B-trees in large databases.}

B-trees are a popular data structure used in large databases and filesystems due to their efficient search, insertion, and deletion operations and their ability to handle large amounts of data. A B-tree is a balanced multiway search tree where each node can have multiple children (as opposed to binary search trees, which have only two children per node).

The key features of B-trees that make them suitable for large databases are:

1. **Balanced structure**: B-trees guarantee a logarithmic height, ensuring that search, insertion, and deletion operations have a time complexity of O(log n). This property is critical for maintaining performance as the size of the database grows.

2. **High fan-out**: Each node in a B-tree can have multiple children, which reduces the height of the tree compared to binary trees. A high branching factor (the number of children per node) allows B-trees to store large amounts of data while keeping the tree height relatively small. This results in fewer disk accesses, which is crucial for large databases stored on disk.

3. **Efficient disk access**: B-trees are designed to minimize disk access since disk I/O is typically much slower than in-memory operations. B-trees are optimized to read or write an entire node (containing multiple keys and pointers) in a single disk access. This reduces the number of disk operations required for search, insertion, and deletion operations.

4. **Support for range queries**: B-trees store keys in a sorted order within each node and across the entire tree, which enables efficient range queries. A range query retrieves all records with keys within a specified range. In a B-tree, this operation can be performed using a single tree traversal.

Some common applications of B-trees in large databases include:

1. **Indexing**: B-trees are often used as index structures in databases, allowing for fast retrieval of records based on their keys. Database systems like MySQL, PostgreSQL, and SQLite use B-trees (or B-tree variants) for their primary indexing mechanism.

2. **Filesystems**: Some filesystems, like the HFS+ (Mac OS) and NTFS (Windows), use B-trees or B-tree variants (like B+ trees) to organize their directory structures and manage file allocation. This ensures efficient file storage and retrieval operations.

3. **NoSQL databases**: B-trees are used in various NoSQL databases, like CouchDB and HBase, which require efficient key-value storage and retrieval. B-trees provide a scalable and efficient solution for these databases.

In summary, B-trees are a critical data structure for large databases due to their balanced structure, high fan-out, efficient disk access, and support for range queries. They are widely used in indexing, filesystems, and NoSQL databases to maintain performance as the size of the data grows.


\section{Graphs and graph algorithms}

The concept of a graph, its representations and visualization. Euler and Hamiltonian paths. Tree graphs. Fundamental graph algorithms. Depth-first and breadth-first search. Optimal spanning trees. Prim’s Kruskal’s algorithm. Shortest paths. Dijkstra and Bellman-Ford algorithms.


\subsection{The concept of a graph, its representations and visualization}

**Graphs:**

A graph is a mathematical structure that models relationships between objects. It consists of a finite set of vertices (also called nodes or points) and a finite set of edges (also called arcs or lines) that connect these vertices. Graphs can be used to represent various real-world scenarios, such as social networks, transportation networks, computer networks, and more.

There are two main types of graphs:

1. **Undirected graphs**: In an undirected graph, edges have no direction, meaning that an edge connecting two vertices can be traversed in both directions. The order of the vertices in an undirected edge does not matter.
2. **Directed graphs (digraphs)**: In a directed graph, edges have a direction, meaning that an edge goes from one vertex to another but not necessarily the other way around. The order of the vertices in a directed edge matters.

**Graph Representations:**

There are several ways to represent a graph in a computer's memory, the most common being adjacency matrices and adjacency lists.

1. **Adjacency Matrix**: An adjacency matrix is a square matrix of size n x n, where n is the number of vertices in the graph. The entry at the ith row and jth column (A[i][j]) represents the weight of the edge between vertices i and j. If there's no edge between vertices i and j, A[i][j] is usually set to 0 or infinity, depending on the context. Adjacency matrices provide fast access to edge information but may consume more space for sparse graphs.

2. **Adjacency List**: An adjacency list is an array of linked lists, where each index in the array corresponds to a vertex in the graph. The linked list at each index contains all the vertices connected to the corresponding vertex. Adjacency lists are more space-efficient for sparse graphs compared to adjacency matrices, but edge access may be slower.

**Graph Visualization:**

Visualizing a graph involves drawing its vertices as points and its edges as lines connecting these points. Common ways to visualize a graph include:

1. **Node-link diagrams**: In a node-link diagram, vertices are represented as circles or other shapes, and edges are represented as lines or curves connecting the vertices. Directed graphs may use arrows to indicate edge direction. Node-link diagrams are the most common visualization method for small to medium-sized graphs.

2. **Matrix view**: In a matrix view, the graph is visualized using its adjacency matrix representation. Rows and columns correspond to vertices, and the cell at the intersection of a row and a column represents the edge between the corresponding vertices. A colored or patterned cell indicates the presence of an edge. Matrix views can be useful for visualizing dense graphs or graphs with a large number of vertices.

Various graph drawing algorithms exist to determine the position of vertices and edges to create visually appealing and easy-to-understand graph visualizations. These algorithms attempt to optimize certain aesthetic criteria, such as minimizing edge crossings, evenly distributing vertices, and maintaining symmetry.

In summary, the concept of a graph involves vertices and edges that represent relationships between objects. Graphs can be represented using adjacency matrices or adjacency lists and can be visualized using node-link diagrams or matrix views. Graph algorithms provide ways to analyze and manipulate graphs to solve various real-world problems.


\subsection{Euler and Hamiltonian paths}

Eulerian and Hamiltonian paths are concepts related to graph traversal, which involve visiting vertices or edges of a graph in a specific sequence. These paths are named after mathematicians Leonhard Euler and William Rowan Hamilton.

**Eulerian Path and Eulerian Circuit:**

An Eulerian path is a traversal of a graph that visits each edge exactly once. If the Eulerian path starts and ends at the same vertex, it's called an Eulerian circuit (or Eulerian cycle). Not all graphs have Eulerian paths or circuits.

A graph has an Eulerian path if and only if it's connected (ignoring isolated vertices) and the number of vertices with odd degrees is either 0 or 2. If there are 0 vertices with odd degrees, the graph has an Eulerian circuit; if there are 2 vertices with odd degrees, the graph has an Eulerian path but not an Eulerian circuit.

Eulerian paths and circuits can be found using Fleury's Algorithm or Hierholzer's Algorithm.

**Hamiltonian Path and Hamiltonian Circuit:**

A Hamiltonian path is a traversal of a graph that visits each vertex exactly once. If the Hamiltonian path starts and ends at the same vertex, it's called a Hamiltonian circuit (or Hamiltonian cycle). Not all graphs have Hamiltonian paths or circuits.

Finding a Hamiltonian path or circuit is an NP-complete problem, meaning there is no known efficient algorithm to solve it for all cases. Some heuristics and algorithms exist for specific types of graphs or to find approximate solutions, such as the backtracking algorithm, the Nearest Neighbor heuristic, and the Christofides algorithm.

**Differences between Eulerian and Hamiltonian Paths:**

1. Eulerian paths visit each edge exactly once, while Hamiltonian paths visit each vertex exactly once.
2. The existence of Eulerian paths and circuits can be determined by the degrees of the vertices, while determining the existence of Hamiltonian paths and circuits is an NP-complete problem with no known efficient algorithm for all cases.
3. Eulerian paths and circuits have well-defined algorithms to find them (Fleury's Algorithm and Hierholzer's Algorithm), while Hamiltonian paths and circuits often rely on heuristics or algorithms for specific cases.

In summary, Eulerian and Hamiltonian paths are graph traversal concepts that involve visiting edges or vertices of a graph in a specific sequence. Eulerian paths visit each edge exactly once, while Hamiltonian paths visit each vertex exactly once. Determining the existence and finding Eulerian paths is relatively straightforward, while Hamiltonian paths are more computationally challenging to identify.


\subsection{Tree graphs}

A tree is a special type of graph that has certain unique properties. In graph theory, a tree is an undirected, connected graph with no cycles. This means that there is exactly one unique path between any two vertices in the tree. Trees are common data structures in computer science and have various applications in algorithms, data management, and network design.

Here are some key properties and terminology related to tree graphs:

1. **Rooted Tree**: A rooted tree is a tree in which one vertex is designated as the root. In a rooted tree, the direction is implied from the root towards the leaves (vertices with no children). Every vertex in a rooted tree, except the root, has a unique parent (the vertex it is connected to towards the root).

2. **Degree**: The degree of a vertex in a tree is the number of edges incident to it. In a rooted tree, the degree of a vertex is equal to the number of children it has plus one (for the edge connecting to its parent, except for the root).

3. **Leaf**: A leaf is a vertex in a tree with a degree of 1 (in a rooted tree, it has no children).

4. **Internal Vertex**: An internal vertex is a vertex in a tree that is not a leaf. In a rooted tree, internal vertices have at least one child.

5. **Height**: The height of a tree is the length of the longest path from the root to a leaf. In a rooted tree, the height of a vertex is the length of the longest path from that vertex to a leaf.

6. **Depth**: The depth of a vertex in a rooted tree is the length of the path from the root to that vertex. The root has a depth of 0.

7. **Subtree**: A subtree is a connected subgraph of a tree that is itself a tree. In a rooted tree, every vertex defines a subtree, which includes the vertex itself and all its descendants.

8. **Binary Tree**: A binary tree is a rooted tree in which each vertex has at most two children, usually referred to as the left child and the right child.

Trees have several applications in computer science and other fields:

1. **Data Structures**: Trees are used to implement various data structures, such as binary search trees, heaps, and trie structures. These data structures provide efficient algorithms for searching, sorting, and managing data.

2. **Hierarchical Relationships**: Trees can be used to represent hierarchical relationships between objects, such as organizational charts, file systems, and taxonomies.

3. **Network Design**: Trees are commonly used in network design, such as the construction of minimum spanning trees in graph algorithms like Kruskal's and Prim's algorithms.

4. **Parsing and Compilation**: Trees are used in parsing algorithms for compilers and interpreters, such as syntax trees and abstract syntax trees, to represent the structure of a program.

In summary, tree graphs are a special type of graph that is undirected, connected, and acyclic. Trees have several unique properties and terminology, such as rooted trees, leaves, internal vertices, and height. They have various applications in computer science, including implementing data structures, representing hierarchical relationships, network design, and parsing algorithms.


\subsection{Optimal spanning trees. Prim’s Kruskal’s algorithm}

Optimal spanning trees, often referred to as minimum spanning trees (MST), are a concept in graph theory related to finding a tree that spans all the vertices in a connected, undirected graph with weighted edges while minimizing the total edge weight. In other words, an MST connects all the vertices in the graph such that the sum of the edge weights is as small as possible, and there are no cycles.

There are several algorithms to find the minimum spanning tree of a graph, with the most common being Kruskal's Algorithm and Prim's Algorithm.

**Kruskal's Algorithm:**

Kruskal's algorithm is a greedy algorithm that works by sorting the edges in the graph by their weights and adding them to the MST one by one as long as they do not create a cycle. The algorithm can be described as follows:

1. Create an empty set for the MST edges.
2. Sort all the edges of the graph by their weight in non-decreasing order.
3. For each edge in the sorted list:
   a. Check if adding this edge would create a cycle in the MST. You can use a disjoint-set data structure (also called a union-find data structure) to efficiently check for cycles.
   b. If adding the edge does not create a cycle, add it to the MST edges set.
4. Continue until the MST contains all the vertices of the graph.

**Prim's Algorithm:**

Prim's algorithm is another greedy algorithm that starts with an arbitrary vertex and grows the MST one edge at a time by selecting the minimum-weight edge that connects a vertex in the MST to a vertex outside the MST. The algorithm can be described as follows:

1. Choose an arbitrary vertex as the starting vertex.
2. Create an empty set for the MST edges and a set containing the starting vertex.
3. While there are vertices not yet in the MST:
   a. Find the edge with the smallest weight that connects a vertex in the MST to a vertex outside the MST.
   b. Add this edge to the MST edges set and the connected vertex to the MST vertices set.
4. Continue until the MST contains all the vertices of the graph.

Both Kruskal's and Prim's algorithms have a time complexity of O(E log E) or O(E log V), where E is the number of edges and V is the number of vertices in the graph. The choice between the two algorithms depends on the specific problem and the implementation details.

Minimum spanning trees have various applications in network design, such as designing efficient transportation networks, telecommunications infrastructure, and electrical power grids, where the goal is to minimize the total cost of connecting all the nodes in the network.


\subsection{Shortest paths. Dijkstra and Bellman-Ford algorithms.}

Shortest path algorithms are used to find the shortest path between two vertices in a graph with weighted edges. The two most common algorithms for solving the single-source shortest path problem are Dijkstra's Algorithm and the Bellman-Ford Algorithm. These algorithms can be applied to directed or undirected graphs and are used in various applications, such as routing in computer networks, navigation systems, and transportation planning.

**Dijkstra's Algorithm:**

Dijkstra's algorithm is a greedy algorithm that finds the shortest path from a source vertex to all other vertices in the graph. It works by iteratively selecting the vertex with the smallest known distance from the source and updating the distances of its neighbors. The algorithm can be described as follows:

1. Assign an initial distance of 0 to the source vertex and infinity (∞) to all other vertices.
2. Create a set of unvisited vertices and a priority queue (e.g., a min-heap) containing the vertices with their initial distances.
3. While there are unvisited vertices:
   a. Extract the vertex with the smallest distance from the priority queue.
   b. Mark the vertex as visited.
   c. For each unvisited neighbor of the current vertex, calculate the distance through the current vertex.
   d. If the calculated distance is less than the known distance for that neighbor, update the neighbor's distance in the priority queue.
4. Continue until all vertices are visited or the target vertex is visited (if looking for a path to a specific vertex).

Dijkstra's algorithm has a time complexity of O(V\^2) for a naive implementation, but it can be improved to O(V log V + E) using a priority queue (e.g., a binary heap or a Fibonacci heap).

Note that Dijkstra's algorithm does not work for graphs with negative edge weights, as it assumes that the shortest path can only be improved by visiting vertices with smaller distances.

**Bellman-Ford Algorithm:**

The Bellman-Ford algorithm is a dynamic programming algorithm that can find the shortest path from a source vertex to all other vertices in a graph, even if it contains negative edge weights. However, it cannot handle graphs with negative cycles. The algorithm works by iteratively updating the distances of all vertices for |V| - 1 iterations, where |V| is the number of vertices in the graph. The algorithm can be described as follows:

1. Assign an initial distance of 0 to the source vertex and infinity (∞) to all other vertices.
2. For each of the |V| - 1 iterations:
   a. For each edge (u, v) in the graph with weight w(u, v), update the distance of vertex v if the distance through vertex u is shorter: `distance[v] = min(distance[v], distance[u] + w(u, v))`
3. After the |V| - 1 iterations, check for negative cycles by iterating through all edges. If any distance can be updated, there is a negative cycle in the graph.

The Bellman-Ford algorithm has a time complexity of O(V * E), which is slower than Dijkstra's algorithm for most graphs. However, it can handle graphs with negative edge weights as long as there are no negative cycles.

In summary, Dijkstra's and Bellman-Ford algorithms are used to find the shortest paths in graphs with weighted edges. Dijkstra's algorithm is faster for graphs without negative edge weights, while the Bellman-Ford algorithm can handle graphs with negative edge weights but cannot handle negative cycles. Both algorithms have numerous applications in computer science and other fields where finding the shortest path between nodes is essential.


\section{Imperative languages, object-oriented programming}

Arithmetical and logical operations. Control structures (conditional and unconditional flow control, loops). Functions and procedures (subprograms, subroutines), data passing, local and global variables, the role of stacks in function calls. Basics of object-oriented programming: classes, objects, inheritance, polymorphism, encapsulation.


\subsection{Arithmetical and logical operations}

Imperative languages and object-oriented programming are programming paradigms that use a sequence of instructions to manipulate data and control the flow of a program. In these paradigms, arithmetic and logical operations are fundamental building blocks for performing calculations and making decisions.

**Arithmetic Operations:**

Arithmetic operations are used to perform mathematical calculations on numeric data types (e.g., integers, floating-point numbers). The basic arithmetic operations include:

1. Addition (`+`): Adds two numbers together.
2. Subtraction (`-`): Subtracts one number from another.
3. Multiplication (`*`): Multiplies two numbers together.
4. Division (`/`): Divides one number by another, often returning a floating-point result.
5. Modulus or Remainder (`%`): Calculates the remainder of the division of one number by another (mostly applicable to integers).

Example in Python:

```python
a = 10
b = 5

addition_result = a + b
subtraction_result = a - b
multiplication_result = a * b
division_result = a / b
modulus_result = a % b
```

**Logical Operations:**

Logical operations are used to perform boolean logic on boolean data types (e.g., `True` or `False`). The basic logical operations include:

1. AND (`&&` or `and`): Returns `True` if both operands are true, otherwise `False`.
2. OR (`||` or `or`): Returns `True` if at least one of the operands is true, otherwise `False`.
3. NOT (`!` or `not`): Returns `True` if the operand is false, and `False` if the operand is true.

Example in Python:

```python
a = True
b = False

and_result = a and b
or_result = a or b
not_result = not a
```

**Comparison Operations:**

Comparison operations are used to compare two values and return a boolean result. These operations are often used in combination with logical operations to create more complex conditions. The basic comparison operations include:

1. Equal to (`==`): Returns `True` if the operands are equal, otherwise `False`.
2. Not equal to (`!=`): Returns `True` if the operands are not equal, otherwise `False`.
3. Less than (`<`): Returns `True` if the left operand is less than the right operand, otherwise `False`.
4. Greater than (`>`): Returns `True` if the left operand is greater than the right operand, otherwise `False`.
5. Less than or equal to (`<=`): Returns `True` if the left operand is less than or equal to the right operand, otherwise `False`.
6. Greater than or equal to (`>=`): Returns `True` if the left operand is greater than or equal to the right operand, otherwise `False`.

Example in Python:

```python
a = 10
b = 5

equal_result = a == b
not_equal_result = a != b
less_than_result = a < b
greater_than_result = a > b
less_than_or_equal_result = a <= b
greater_than_or_equal_result = a >= b
```

In summary, arithmetic, logical, and comparison operations are fundamental components of imperative languages and object-oriented programming. They enable developers to perform calculations, make decisions, and compare values within their programs. These operations are essential for controlling the flow of a program and manipulating data to achieve the desired output.


\subsection{Control structures (conditional and unconditional flow control, loops)}

Control structures are essential components of imperative and object-oriented programming languages, as they help manage the flow of a program. Common control structures include conditional and unconditional flow control and loops.

**Conditional Flow Control:**

Conditional flow control structures allow the execution of a block of code based on a specific condition. The most common conditional flow control structures are the `if`, `else if`, and `else` statements. They evaluate a condition (usually involving a comparison operation) and execute the corresponding block of code if the condition is true.

Example in Python:

```python
age = 18

if age >= 21:
    print("You can legally drink alcohol.")
elif age >= 18:
    print("You can legally vote, but you cannot drink alcohol.")
else:
    print("You cannot legally vote or drink alcohol.")
```

**Unconditional Flow Control:**

Unconditional flow control structures alter the program's flow without evaluating a specific condition. The most common unconditional flow control structure is the `goto` statement, which transfers the control to a labeled statement in the code (note that `goto` is not available in some modern programming languages, like Python, due to its potential to create unreadable and unmanageable code).

Another example of unconditional flow control is the `break` statement, which is used inside loops to exit the loop immediately, skipping the remaining iterations.

Example in Python (using `break`):

```python
for i in range(1, 11):
    if i == 5:
        break
    print(i)
```

**Loops:**

Loops are control structures that enable the repeated execution of a block of code until a specific condition is met. There are two main types of loops: `for` loops and `while` loops.

1. `for` loops: A `for` loop iterates over a sequence (e.g., a list, tuple, or range) and executes a block of code for each element in the sequence. In Python, `for` loops work with any iterable object.

Example in Python:

```python
for i in range(5):
    print(i)
```

2. `while` loops: A `while` loop repeatedly executes a block of code as long as a given condition is true. The condition is evaluated before each iteration, and if it is false, the loop is terminated.

Example in Python:

```python
count = 0

while count < 5:
    print(count)
    count += 1
```

Control structures play a critical role in managing the flow of a program, allowing developers to create more dynamic and flexible code. Conditional and unconditional flow control structures help make decisions based on conditions, while loops enable the repeated execution of code blocks. Together, these control structures form the foundation of imperative and object-oriented programming languages.


\subsection{Functions and procedures (subprograms, subroutines)}

Functions and procedures, also known as subprograms or subroutines, are fundamental components in imperative and object-oriented programming languages. They are reusable blocks of code that perform a specific task, allowing for better organization, modularity, and maintainability of a program. Using functions and procedures enables code reuse, which reduces code redundancy and makes the overall program easier to understand and modify.

**Functions:**

Functions are subprograms that take input parameters, perform a specific task, and return a value. Functions can be called by providing the required parameters and can be used to compute a result or process data. Functions can be defined by the programmer or built-in as part of a programming language's standard library.

Example in Python:

```python
def add_numbers(a, b):
    return a + b

result = add_numbers(5, 3)
print(result)  # Output: 8
```

**Procedures:**

Procedures, also known as subroutines, are similar to functions, but they do not return a value. Instead, they perform a specific task or a series of tasks, usually involving side effects (e.g., modifying variables, printing output). Procedures are used for code organization and modularity when a return value is not necessary.

In some programming languages, procedures and functions are distinct constructs. However, in Python and many other modern languages, the distinction between functions and procedures is less clear, as functions can be used as procedures by simply not returning a value.

Example in Python (a function used as a procedure):

```python
def print_greeting(name):
    print(f"Hello, {name}!")

print_greeting("Alice")  # Output: Hello, Alice!
```

**Parameters and Arguments:**

Functions and procedures can accept input values, called parameters, which allow for the passing of data into the subprogram. When a function or procedure is called with specific values for these parameters, the values are referred to as arguments.

Parameters can be passed by value (where a copy of the value is passed) or by reference (where a reference to the memory location of the original value is passed). The method of passing parameters depends on the programming language and the data type of the parameter. In Python, for example, mutable objects like lists and dictionaries are passed by reference, while immutable objects like numbers and strings are passed by value.

Functions and procedures are essential components of imperative and object-oriented programming languages, as they provide a modular and reusable structure for organizing code. They improve code readability, maintainability, and reusability, making complex programs more manageable and easier to understand.


\subsection{data passing, local and global variables}

In imperative and object-oriented programming languages, data passing, local variables, and global variables are important concepts that enable efficient communication between different parts of a program.

**Data Passing:**

Data passing refers to the process of transferring data between different parts of a program, such as between functions or procedures. There are two primary ways to pass data: pass by value and pass by reference.

1. Pass by Value: When data is passed by value, a copy of the original value is sent to the function or procedure. Any changes made to the copy inside the function or procedure do not affect the original value.

2. Pass by Reference: When data is passed by reference, a reference to the memory location of the original value is sent to the function or procedure. Any changes made to the value inside the function or procedure directly affect the original value.

In Python, the method of passing data depends on the data type. Immutable objects (e.g., numbers, strings, and tuples) are passed by value, while mutable objects (e.g., lists, dictionaries, and sets) are passed by reference.

**Local Variables:**

Local variables are variables that are declared inside a function, procedure, or code block. They have a local scope, which means that they are only accessible within the block of code where they are declared. When the function or procedure finishes executing, all local variables are destroyed, and their memory is deallocated.

Example in Python:

```python
def example_function():
    local_variable = 10
    print(local_variable)  # Output: 10

example_function()
print(local_variable)  # This will raise an error because local_variable is not defined in this scope
```

**Global Variables:**

Global variables are variables that are declared outside any function or procedure. They have a global scope, which means that they can be accessed from any part of the program. However, special care must be taken when modifying global variables from within a function or procedure, as it can lead to unintended side effects and make the program harder to understand and maintain.

In Python, you can use the `global` keyword to indicate that a variable inside a function or procedure should be treated as a global variable.

Example in Python:

```python
global_variable = 20

def example_function():
    global global_variable
    global_variable = 30
    print(global_variable)  # Output: 30

example_function()
print(global_variable)  # Output: 30
```

Data passing, local variables, and global variables are essential concepts in imperative and object-oriented programming languages, as they enable efficient communication between different parts of a program and provide a structure for managing data. Understanding these concepts is crucial for writing clean, efficient, and maintainable code.


\subsection{the role of stacks in function calls}

In computer programming, the stack is a data structure that plays a critical role in the execution of function calls. When a function is called, the program uses a stack to manage the function's execution context, which includes local variables, function parameters, and return addresses. The stack is a last-in, first-out (LIFO) structure, which means that the most recently added item is the first one to be removed.

Here are some essential aspects of stacks in function calls:

**Stack Frame:**

When a function is called, a new stack frame is created and pushed onto the stack. A stack frame is a memory block that contains the function's execution context, including:

1. Function parameters: The values (or references) passed as arguments to the function.
2. Local variables: The variables declared inside the function.
3. Return address: The memory address to which the program should return after the function finishes executing.

**Function Call:**

When a function is called, the following steps occur:

1. The program pushes the return address (the next instruction after the function call) onto the stack.
2. The program pushes the function's parameters onto the stack.
3. The program transfers control to the function by jumping to its starting memory address.
4. Inside the function, a new stack frame is created, and local variables are allocated within this frame.
5. The function executes its code, using the stack to manage its local variables and parameters.

**Function Return:**

When a function finishes executing and is ready to return, the following steps occur:

1. If the function has a return value, it is typically placed in a specific register or memory location for the calling function to access.
2. The function's stack frame is popped from the stack, and the local variables and parameters are deallocated.
3. The program retrieves the return address from the stack and jumps back to that address, resuming execution in the calling function.

**Recursion:**

Recursion occurs when a function calls itself, either directly or indirectly through another function. The stack plays a crucial role in managing recursive function calls, as each call creates a new stack frame with its own set of local variables and parameters. This allows each recursive call to maintain its own execution context without interfering with other calls. However, excessive recursion can lead to a stack overflow, which occurs when the stack's memory is exhausted, causing the program to crash.

In summary, the stack is an essential data structure in function calls, managing the execution context, local variables, function parameters, and return addresses. The stack enables the efficient and organized execution of functions, including the management of recursive calls. Understanding the role of stacks in function calls is vital for grasping the inner workings of programming languages and writing efficient, reliable code.


\subsection{Basics of object-oriented programming: classes, objects, inheritance, polymorphism, encapsulation}

Object-oriented programming (OOP) is a programming paradigm that emphasizes the use of objects, which are instances of classes, to model and organize code. OOP focuses on the relationships between objects and the actions they can perform, making it easier to structure complex programs. The four main principles of OOP are classes, objects, inheritance, polymorphism, and encapsulation.

**Classes:**

A class is a blueprint or template that defines the structure and behavior of objects of the same type. Classes specify the attributes (data) and methods (functions) shared by all instances of the class.

Example in Python:

```python
class Dog:
    def __init__(self, name, breed):
        self.name = name
        self.breed = breed

    def bark(self):
        print("Woof!")
```

**Objects:**

Objects are instances of classes. They are individual entities that have their own state (attributes) and behavior (methods). The state of an object is determined by the values of its attributes, while its behavior is defined by the methods of its class.

Example in Python:

```python
dog1 = Dog("Fido", "Labrador")
dog2 = Dog("Buddy", "Golden Retriever")

dog1.bark()  # Output: Woof!
dog2.bark()  # Output: Woof!
```

**Inheritance:**

Inheritance is a mechanism that allows one class (the subclass or derived class) to inherit the attributes and methods of another class (the superclass or base class). This promotes code reusability and modularity, as common functionality can be defined in a base class and extended or overridden in derived classes.

Example in Python:

```python
class Animal:
    def speak(self):
        pass

class Dog(Animal):
    def speak(self):
        print("Woof!")

class Cat(Animal):
    def speak(self):
        print("Meow!")

dog = Dog()
cat = Cat()

dog.speak()  # Output: Woof!
cat.speak()  # Output: Meow!
```

**Polymorphism:**

Polymorphism is the ability of a single interface or function to represent different types or classes. It allows objects of different classes to be treated as objects of a common superclass, enabling the same function or method call to produce different results based on the object's class.

Example in Python (using the previous `Animal`, `Dog`, and `Cat` classes):

```python
def make_animal_speak(animal):
    animal.speak()

dog = Dog()
cat = Cat()

make_animal_speak(dog)  # Output: Woof!
make_animal_speak(cat)  # Output: Meow!
```

**Encapsulation:**

Encapsulation is the process of hiding the internal details of an object and exposing only what is necessary. This is achieved by using private attributes (data) and public methods (functions) in a class, which helps protect the object's state from unintended modifications and ensures that the object can only be interacted with through its public interface.

Example in Python:

```python
class BankAccount:
    def __init__(self, balance):
        self.__balance = balance

    def deposit(self, amount):
        self.__balance += amount

    def withdraw(self, amount):
        if amount <= self.__balance:
            self.__balance -= amount
        else:
            print("Insufficient funds")

    def get_balance(self):
        return self.__balance

account = BankAccount(100)
account.deposit(50)
account.withdraw(25)
print(account.get_balance())  # Output: 125
```

Object-oriented programming provides a structured way to organize code by focusing on objects and their interactions. The principles of classes, objects, inheritance, polymorphism, and encapsulation enable the creation of modular, reusable, and maintainable code, simplifying the development of complex programs.


\section{Compilers}

Chomsky classification of formal languages, regular languages, their properties and relation to lexical analyzer programs. Finite automata. The structure of compilers, the process of compilation or interpretation, look-ahead. Context-free grammars, their properties and role in the theory of compilers. Methods of parsing (LL(k), LR(1), grammar, syntax trees, etc.) Implementation of symbol trees and operations on them.


\subsection{Chomsky Classification}

https://zh.wikipedia.org/wiki/%E4%B9%94%E5%A7%86%E6%96%AF%E5%9F%BA%E8%B0%B1%E7%B3%BB


\subsection{Regular Languages}

https://zh.wikipedia.org/zh-cn/%E6%AD%A3%E5%88%99%E8%AF%AD%E8%A8%80


\subsection{relation to lexical analyzer programs}

Lexical analyzers, also known as lexers or scanners, are programs used in the early stages of the compilation or interpretation process. They are responsible for processing a sequence of input symbols (usually source code) and breaking it down into a sequence of tokens, each representing a meaningful unit of the programming language. These tokens are then passed on to a parser, which is responsible for checking the syntactic structure of the input and generating an abstract syntax tree (AST) or similar data structure.

The relation between regular languages and lexical analyzers lies in the fact that regular languages can be used to describe the patterns and rules that define the tokens of a programming language. Regular expressions, which are a concise way to represent regular languages, can be used to define the structure of tokens, such as keywords, identifiers, literals, and various operators in a programming language.

Lexical analyzers use finite automata (either deterministic or non-deterministic) to efficiently recognize regular languages and match input symbols to the corresponding regular expressions. These finite automata can be constructed from the regular expressions that define the tokens of the programming language.

In practice, many lexer generators and tools, such as Lex, Flex, or ANTLR, exist to assist in creating lexical analyzers. These tools typically take regular expressions as input, generate the corresponding finite automata, and produce code for a lexical analyzer that can tokenize input according to the specified regular expressions.

In summary, regular languages are an essential concept in the design and implementation of lexical analyzers. They provide a way to define the structure of tokens in a programming language and enable the efficient recognition and tokenization of input symbols using finite automata.


\subsection{AST}

https://juejin.cn/post/6844904035271573511


\subsection{Finite automata}

A finite automaton (FA), also known as a finite-state machine (FSM), is a theoretical model of computation used to recognize patterns and represent sequential logic. It consists of a finite set of states, a finite set of input symbols, a transition function, an initial state, and a set of accept states. Finite automata are used to recognize regular languages, which are the simplest class of languages in the Chomsky Hierarchy.

There are two main types of finite automata:

1. **Deterministic Finite Automaton (DFA)**: For each state and input symbol, there is exactly one transition to another state (or possibly the same state). In other words, given the current state and input symbol, the next state is uniquely determined.

2. **Nondeterministic Finite Automaton (NFA)**: For each state and input symbol, there can be multiple transitions to different states, or even none. Additionally, NFAs may have ε-transitions, which allow transitioning between states without consuming any input symbols.

Both DFA and NFA have equivalent expressive power, meaning they can recognize the same set of regular languages. However, NFAs can sometimes be more compact and easier to construct. There are algorithms, such as the powerset construction, to convert an NFA into an equivalent DFA.

Finite automata have various applications in computer science, including:

* **Lexical analysis**: As mentioned earlier, finite automata are used in lexical analyzers to tokenize input symbols (e.g., source code) based on regular expressions that define the tokens of a programming language.

* **Pattern matching**: Finite automata can be used to search for patterns in text, such as in grep or other text search utilities. Regular expressions are often used to define the search patterns, and finite automata perform the actual pattern matching.

* **Protocol verification**: Finite automata can model the behavior of communication protocols and verify their correctness by exploring the possible states and transitions.

* **Control systems**: Finite automata can be used to model the behavior of digital circuits and control systems, which can help in the design and verification of these systems.

In summary, finite automata are abstract computational models that can recognize regular languages and represent sequential logic. They have numerous applications in computer science, including lexical analysis, pattern matching, protocol verification, and control systems. Both deterministic and nondeterministic finite automata have equivalent expressive power, but NFAs can sometimes be more compact and easier to construct.


\subsection{The structure of compilers, the process of compilation or interpretation, look-ahead}

## Structure of Compilers

A compiler is a software program that translates high-level source code written in a programming language into low-level machine code or an intermediate representation that can be executed by a computer. The process of compilation involves several stages, including lexical analysis, syntax analysis, semantic analysis, intermediate code generation, optimization, and code generation. Here's a brief overview of these stages:

1. **Lexical analysis**: The first phase of compilation, in which the source code is broken down into a sequence of tokens, including keywords, identifiers, literals, and symbols.

2. **Syntax analysis**: Also known as parsing, this phase involves organizing the tokens into a hierarchical structure called an abstract syntax tree (AST) based on the grammar of the language.

3. **Semantic analysis**: In this phase, the compiler checks the AST for semantic correctness, such as type-checking, symbol resolution, and scope analysis.

4. **Intermediate code generation**: The compiler converts the AST into an intermediate representation (IR), which is a lower-level, platform-independent representation of the code.

5. **Optimization**: The compiler optimizes the IR to improve the performance and efficiency of the generated code.

6. **Code generation**: The final phase of compilation, in which the optimized IR is converted into the target machine code or assembly language.

## Process of Compilation or Interpretation

Compilation and interpretation are two methods of executing code written in a programming language. Here's a brief comparison of the two:

- **Compilation**: In this process, the source code is translated into machine code before execution. The compiled code is saved as a separate file, usually an executable or object file, which can be executed by the target machine without the need for the compiler. Compilation is generally slower than interpretation, but the resulting executable runs faster since the code has already been translated to machine code.

- **Interpretation**: An interpreter directly executes the source code without translating it to machine code. The interpreter reads and executes the code line-by-line during runtime. Interpretation is generally slower than compilation since the code is being translated and executed on-the-fly, but it offers more flexibility and easier debugging.

Some programming languages, such as Python and Java, use a combination of both techniques, where the source code is compiled to an intermediate representation called bytecode, which is then interpreted by a virtual machine.

## Look-Ahead

Look-ahead is a technique used in compilers, specifically during the parsing phase, to predict what comes next in the input stream based on the current state of the parser. It helps the parser make decisions about which production rule to apply in case of ambiguity.

There are different types of look-ahead techniques:

- **LL(k)**: This is a top-down parsing technique where the parser looks ahead by `k` tokens to make a decision. LL stands for "Left-to-right, Leftmost derivation," and the `k` represents the number of tokens the parser can look ahead.

- **LR(k)**: This is a bottom-up parsing technique where the parser looks ahead by `k` tokens to make a decision. LR stands for "Left-to-right, Rightmost derivation," and the `k` represents the number of tokens the parser can look ahead.

Increasing the look-ahead can improve the parser's capability to handle more complex grammar rules and resolve ambiguities, but it also increases the complexity of the parsing algorithm.


\subsection{Methods of parsing (LL(k), LR(1), grammar, syntax trees, etc.)}

Parsing, also known as syntax analysis, is the process of analyzing a sequence of tokens to determine their grammatical structure according to a formal grammar. There are several parsing techniques used in compilers, and we'll discuss some of the most common ones, including LL(k), LR(1), and their associated concepts such as grammar and syntax trees.

### LL(k) Parsing

LL(k) parsing is a top-down, predictive parsing technique that uses a left-to-right scanning of the input and constructs a leftmost derivation. The 'k' in LL(k) refers to the look-ahead of k tokens used to make decisions during parsing.

This method starts with the start symbol of the grammar and expands the non-terminals using the production rules until a match with the input is found. LL(k) parsers use a recursive descent algorithm or a table-driven approach to implement parsing.

### LR(1) Parsing

LR(1) parsing is a bottom-up technique that constructs a rightmost derivation in reverse order. The '1' in LR(1) refers to the look-ahead of one token used to make decisions during parsing. LR(1) parsers are more powerful than LL(k) parsers and can handle a larger class of grammars.

LR(1) parsing involves constructing a parse table using the grammar rules and the LR(1) items, which consist of a production rule, a position marker (dot), and a lookahead token. The parsing process uses a stack to hold the parser states, and the decisions are made based on the current state, lookahead token, and the parse table entries.

There are several types of LR parsers, such as Simple LR (SLR), Canonical LR, and Lookahead LR (LALR), which differ in the way the parse table is constructed and the number of parser states.

### Grammar

A grammar is a set of production rules that define the syntax of a programming language. It's typically expressed using a notation called Backus-Naur Form (BNF) or Extended Backus-Naur Form (EBNF). A grammar consists of:

- **Non-terminal symbols**: Abstract syntactic categories that represent a group of related structures.
- **Terminal symbols**: Tokens that correspond to the leaf nodes in the syntax tree (keywords, identifiers, literals, etc.).
- **Production rules**: Rules that define how non-terminal symbols can be replaced by a sequence of terminal and non-terminal symbols.

### Syntax Trees

A syntax tree, also known as a parse tree or an abstract syntax tree (AST), is a hierarchical representation of the source code's grammatical structure. The tree's internal nodes represent non-terminal symbols, while the leaf nodes represent terminal symbols. Syntax trees are generated during the parsing phase and are used in subsequent compiler phases, such as semantic analysis and intermediate code generation.


\subsection{Implementation of symbol trees and operations on them}
Symbol trees, also known as abstract syntax trees (ASTs), are tree-like data structures that represent the hierarchical structure of source code according to a language's grammar. They are generated during the parsing phase of compilation and are used in subsequent phases for semantic analysis, optimization, and code generation. In this section, we'll discuss the implementation of ASTs and some common operations performed on them.

### Implementation

To implement an AST, you first need to define the node types to represent the different language constructs, such as expressions, statements, and declarations. These nodes can be implemented using classes or structs, depending on the programming language. Each node type should have a unique identifier, along with any relevant attributes and child nodes.

Here's a simple example in Python:

```python
class Node:
    def __init__(self, node_type, children=None, **attributes):
        self.node_type = node_type
        self.children = children or []
        self.attributes = attributes

    def add_child(self, child):
        self.children.append(child)
```

With this basic implementation, you can create nodes for different language constructs, such as:

```python
# Variable declaration
var_decl = Node("VAR_DECL", identifier="x", data_type="int")

# Assignment statement
assign_stmt = Node("ASSIGN_STMT", identifier="x", expression=Node("LITERAL", value=42))
```

### Common Operations

Here are some common operations that can be performed on ASTs:

1. **Traversal**: Traversing the AST is a fundamental operation, as it allows you to visit each node in the tree to perform various analyses or transformations. Common tree traversal algorithms include depth-first search (DFS) and breadth-first search (BFS). You can implement these traversals using either recursion or iteration.

2. **Pretty-printing**: This operation involves converting the AST back to a human-readable representation of the source code. This can be useful for debugging or code formatting purposes. You can implement pretty-printing using a recursive traversal of the AST, generating the corresponding source code for each node.

3. **Semantic analysis**: Once an AST is generated, you can perform various analyses to check for semantic correctness, such as type checking, symbol resolution, and scope analysis. This typically involves traversing the AST and maintaining a symbol table to store information about identifiers and their corresponding declarations.

4. **AST transformations**: You can modify the AST to perform optimizations or code transformations, such as constant folding, dead code elimination, or loop unrolling. This typically involves traversing the AST, identifying patterns or opportunities for optimization, and modifying the tree accordingly.

5. **Code generation**: The final phase of compilation involves converting the AST to target machine code or an intermediate representation, such as bytecode or LLVM IR. This can be accomplished by traversing the AST and generating the corresponding code for each node, using a stack-based or register-based approach.



\section{System engineering and software development}

Basics of UML. Meta-models. The 4-layer meta-model of UML. Basics of workflow modeling. The description of the steps of the classic and RUP methodologies with the Software Process Engineering Model (SPEM) UML profile. Programming and software engineering technologies and methodologies. Procedural techniques. The elements of the RUP methodology, phases and activities.


\subsection{Basics of UML}

Unified Modeling Language (UML) is a standardized visual language used to model software systems, including their structure, behavior, and interactions. UML provides a set of graphical notation techniques to create abstract models of specific systems, which can then be used as blueprints for actual software development.

Here are the basic building blocks of UML:

1. **UML Diagrams**: UML is comprised of various types of diagrams, each serving a specific purpose. The main diagrams are:

   - Class Diagram
   - Object Diagram
   - Use Case Diagram
   - Sequence Diagram
   - Collaboration Diagram
   - Statechart Diagram
   - Activity Diagram
   - Component Diagram
   - Deployment Diagram

2. **UML Elements**: Each UML diagram consists of several elements, which represent the various aspects of a software system. Some common UML elements are:

   - Classes and Objects
   - Interfaces
   - Relationships (association, aggregation, composition, inheritance, and dependency)
   - Packages
   - Actors
   - Use Cases
   - States and Transitions
   - Activities and Actions
   - Components and Nodes

Let's briefly explain some of the most important UML diagrams:

- **Class Diagram**: This diagram shows the static structure of a system by depicting its classes, attributes, operations, and relationships between objects. It represents the fundamental blueprint of a software system.

- **Object Diagram**: This diagram represents the instances of classes and their relationships at a particular point in time. It is used to visualize the structure of objects and their relationships in the system.

- **Use Case Diagram**: This diagram models the interactions between actors (end-users or other systems) and the system itself. It helps identify the system's functionality and the roles that interact with it.

- **Sequence Diagram**: This diagram represents the dynamic behavior of a system by showing the interactions between objects in a specific sequence. It highlights the order in which messages are exchanged between objects to accomplish a specific task.

- **Statechart Diagram**: This diagram models the dynamic behavior of an object by showing its possible states and the transitions between them. It is useful for representing the life cycle of an object over time.

- **Activity Diagram**: This diagram models the flow of control within a system, showing the sequence of activities and decisions involved in the execution of a specific process or use case.

- **Component Diagram**: This diagram represents the physical organization of a software system, displaying the components that make up the system and their dependencies.

- **Deployment Diagram**: This diagram models the hardware architecture of a system, depicting the nodes (hardware devices) and the components that are deployed on them.

These are just the basics of UML. To fully understand and utilize UML, it is essential to learn the various diagrams and elements in detail and practice creating models for real-world software systems.


\subsection{Meta-models. The 4-layer meta-model of UML}

A meta-model is a model that describes the structure, semantics, and constraints of other models. In other words, it is a model of models. Meta-models are used to define the rules and conventions of modeling languages, ensuring that the models created are consistent and adhere to a set of predefined guidelines.

In the context of UML (Unified Modeling Language), the UML meta-model is a specification that defines the structure and semantics of UML diagrams and their elements. The UML meta-model acts as a blueprint for creating UML models that are both consistent and compliant with the UML standards.

There are typically four layers in the meta-modeling hierarchy:

1. **M0 - Instance Layer**: This layer represents the actual instances or objects in the real-world system being modeled.
2. **M1 - Model Layer**: This layer contains the models that describe the structure and behavior of the system. These models are created using the modeling language defined by the meta-model in the M2 layer.
3. **M2 - Meta-Model Layer**: This layer contains the meta-model that defines the structure, semantics, and constraints of the modeling language used in the M1 layer.
4. **M3 - Meta-Meta-Model Layer**: This layer contains the meta-meta-model that describes the structure and semantics of the meta-models in the M2 layer. The most widely known meta-meta-model is the MOF (Meta-Object Facility) standard, which is used to define the structure of various meta-models, including the UML meta-model.

In summary, meta-models are essential for establishing the structure and rules of modeling languages. They provide the foundation for creating consistent and compliant models that accurately represent the systems being modeled. Meta-modeling is a crucial aspect of model-driven engineering and software development methodologies.


\subsection{Basics of workflow modeling}
\subsection{Basics of workflow modeling.}

Workflow modeling is a technique used to represent the sequence of activities, tasks, and decisions involved in a business process or system. The goal of workflow modeling is to analyze and optimize these processes, improve efficiency, and facilitate communication among team members.

Here are the basics of workflow modeling:

1. **Identify the Process**: Determine the process you want to model. It could be a high-level business process or a more specific, low-level process within a larger system. Make sure you clearly define the scope and purpose of the process.

2. **Define Activities**: Break down the process into individual activities or tasks. Activities represent the work performed in the process, such as creating a document, approving a request, or updating a database.

3. **Determine Sequence**: Establish the order in which activities must be executed. Some activities may need to be completed before others can begin, while others may be performed concurrently.

4. **Identify Roles**: Assign roles (or actors) to each activity. Roles can represent people, teams, or systems responsible for performing the activities. Clearly defining roles helps to establish accountability and improve communication within the process.

5. **Model Decisions and Branching**: Identify decision points in the process where the workflow may follow different paths based on certain conditions. These decision points may involve branching (splitting the flow into multiple paths) or merging (combining multiple paths back into a single path).

6. **Add Workflow Elements**: In addition to activities and decision points, there are several other elements commonly used in workflow modeling, such as:

   - Start and end events: Indicate the beginning and end of the process.
   - Gateways: Represent decision points or points where the flow splits or merges.
   - Connectors: Show the flow between activities, events, and gateways.

7. **Choose a Notation**: Select a notation or modeling language to visually represent your workflow model. Some popular notations include Business Process Model and Notation (BPMN), UML Activity Diagrams, and flowcharts. Each notation has its own set of symbols and conventions for representing workflows.

8. **Create the Workflow Diagram**: Using the chosen notation, create a visual representation of the workflow by arranging the elements on a canvas or diagramming tool. Make sure the diagram is clear, easy to understand, and accurately represents the process being modeled.

9. **Validate and Refine**: Review the workflow model with stakeholders and experts to ensure that it accurately represents the process and meets the intended objectives. Refine the model as needed based on feedback and analysis.

10. **Monitor and Improve**: Once the workflow model is implemented, monitor its performance and gather data on areas that may need improvement. Continuously refine the model to optimize the process and adapt to changing requirements.

By following these basic steps, you can create a workflow model that represents and optimizes business processes, leading to improved efficiency, better communication, and more effective decision-making.



\subsection{The description of the steps of the classic and RUP methodologies with the Software Process Engineering Model (SPEM) UML profile}

The Software Process Engineering Meta-model (SPEM) is a UML profile that provides a standardized way to model, define, and manage software development processes. It allows organizations to customize and adapt methodologies like the classic Waterfall model and Rational Unified Process (RUP) to their specific needs.

Let's describe the steps of the classic Waterfall model and RUP using SPEM concepts:

**Classic Waterfall Model**

The Waterfall model is a linear and sequential approach to software development. It consists of distinct phases, with each phase being completed before moving on to the next.

1. *Requirements*: In this phase, the project's functional and non-functional requirements are gathered and documented. In SPEM, this phase can be represented using a **Task Definition** called "Gather Requirements," assigned to a **Role Definition** such as "Business Analyst."

2. *Design*: The system architecture and design are created based on the requirements. In SPEM, this phase can be represented by a Task Definition called "System Design," assigned to a Role Definition such as "System Architect."

3. *Implementation*: The design is translated into source code by developers. In SPEM, this phase can be represented by a Task Definition called "Implement System," assigned to a Role Definition such as "Software Developer."

4. *Testing*: The system is tested to ensure it meets the requirements and is free of defects. In SPEM, this phase can be represented by a Task Definition called "Test System," assigned to a Role Definition such as "Test Engineer."

5. *Deployment*: The system is deployed to the production environment and made available to end-users. In SPEM, this phase can be represented by a Task Definition called "Deploy System," assigned to a Role Definition such as "Deployment Engineer."

6. *Maintenance*: The system is maintained and updated as needed. In SPEM, this phase can be represented by a Task Definition called "Maintain System," assigned to a Role Definition such as "Maintenance Engineer."

**Rational Unified Process (RUP)**

RUP is an iterative and incremental software development process, organized into four phases, each consisting of several iterations.

1. *Inception*: The project's scope, vision, and business case are established. In SPEM, this phase can include Task Definitions such as "Define Project Vision" and "Create Business Case," assigned to Role Definitions like "Project Manager" and "Business Analyst."

2. *Elaboration*: The system's architecture and high-level design are defined, and risks are identified and mitigated. In SPEM, this phase can include Task Definitions like "Define System Architecture," "Identify Risks," and "Develop Risk Mitigation Plan," assigned to Role Definitions such as "System Architect" and "Risk Manager."

3. *Construction*: The system is incrementally developed, integrated, and tested. In SPEM, this phase can include Task Definitions like "Develop Features," "Integrate Components," and "Test Increment," assigned to Role Definitions like "Software Developer" and "Test Engineer."

4. *Transition*: The system is deployed, and end-users are trained and supported. In SPEM, this phase can include Task Definitions like "Deploy System," "Train Users," and "Provide Support," assigned to Role Definitions like "Deployment Engineer" and "Training Specialist."

In both methodologies, SPEM **Work Product Definitions** can be used to represent the artifacts produced during the process, such as requirements documents, design documents, source code, test plans, and user manuals. 

By using the SPEM UML profile, organizations can model their software development processes and customize methodologies like the Waterfall model and RUP to better fit their specific needs and context.


\subsection{Programming and software engineering technologies and methodologies}

Programming and software engineering technologies and methodologies are the tools, techniques, and approaches used to develop, manage, and maintain software systems. They are designed to improve the quality, efficiency, and effectiveness of the software development process. Here's a list of some commonly used technologies and methodologies in software engineering:

**Programming Languages**

These are the languages used to write software programs. Some popular programming languages include:

1. Python
2. Java
3. JavaScript
4. C++
5. C#
6. Ruby
7. PHP
8. Swift
9. Kotlin
10. TypeScript

**Integrated Development Environments (IDEs)**

These are software applications that provide a comprehensive set of tools for software development, such as code editing, debugging, and version control integration. Some popular IDEs are:

1. Visual Studio Code
2. IntelliJ IDEA
3. Eclipse
4. Xcode
5. PyCharm
6. Android Studio
7. NetBeans
8. Sublime Text
9. Atom

**Version Control Systems**

These systems help manage and track changes to software code, allowing developers to collaborate effectively and maintain a history of modifications. Some popular version control systems are:

1. Git
2. Subversion (SVN)
3. Mercurial
4. Perforce
5. Concurrent Versions System (CVS)

**Software Development Methodologies**

These are structured approaches to software development that help manage and organize the process. Some popular methodologies include:

1. Agile (e.g., Scrum, Kanban, Extreme Programming)
2. Waterfall
3. Rational Unified Process (RUP)
4. Spiral
5. V-Model
6. DevOps
7. Lean Software Development

**Project Management and Collaboration Tools**

These tools help plan, coordinate, and track software development projects, facilitating communication among team members. Some popular tools are:

1. Jira
2. Trello
3. Asana
4. Microsoft Project
5. Basecamp
6. Monday.com
7. ClickUp
8. GitHub Projects

**Automated Testing and Continuous Integration Tools**

These tools help automate testing and integration processes, improving the quality and efficiency of software development. Some popular tools include:

1. Jenkins
2. Travis CI
3. TeamCity
4. CircleCI
5. GitLab CI/CD
6. Selenium
7. JUnit
8. TestNG

**Software Design Patterns**

These are reusable solutions to common problems encountered during software design. Some popular design patterns include:

1. Singleton
2. Factory Method
3. Abstract Factory
4. Observer
5. Strategy
6. Decorator
7. Adapter
8. Facade

**Software Architectures**

These are high-level structures and patterns used to organize software systems. Some popular architectures include:

1. Monolithic
2. Microservices
3. Service-Oriented Architecture (SOA)
4. Layered Architecture
5. Event-Driven Architecture
6. Model-View-Controller (MVC)
7. Model-View-ViewModel (MVVM)

These technologies and methodologies are just a starting point. As a software engineer, it's essential to stay up-to-date with the latest trends and best practices in the industry to continue improving and delivering high-quality software systems.


\subsection{Procedural techniques}

Procedural techniques are a programming paradigm that focuses on using procedures (also known as routines, subroutines, or functions) to solve problems. The main idea behind procedural programming is to break down a complex problem into smaller, more manageable tasks. These tasks are then implemented as procedures, which can be called and reused throughout the program. Procedural programming is typically contrasted with object-oriented programming, which focuses on organizing code around objects and their interactions.

Here are some key features of procedural programming:

1. **Sequential Execution**: Procedural programming follows a top-down approach, with code executed line by line in a sequential manner.

2. **Procedures**: The code is organized into reusable procedures, which can be called by other parts of the program. Procedures are typically defined with a specific input (parameters) and output (return value), and they may have local variables.

3. **Modular Design**: Procedural programming promotes modular design, where each procedure is responsible for a specific task. This approach makes it easier to understand, maintain, and modify the code.

4. **Control Structures**: Procedural programming languages provide control structures such as loops (for, while) and conditional statements (if-else) to control the flow of execution.

5. **Global and Local Variables**: Variables in procedural programming can be global (accessible throughout the entire program) or local (accessible only within the procedure in which they are defined). Local variables help to limit the scope of variables and reduce the risk of unintended side effects.

Some popular procedural programming languages include:

1. C
2. Pascal
3. Fortran
4. COBOL
5. BASIC

Many modern programming languages, such as Python, JavaScript, and Java, also support procedural programming techniques alongside other paradigms like object-oriented and functional programming. This allows programmers to choose the best approach for a given problem or combine paradigms as needed.

To use procedural techniques effectively, keep the following principles in mind:

- Break down complex problems into smaller, more manageable tasks.
- Encapsulate tasks within well-defined procedures with clear inputs and outputs.
- Minimize the use of global variables and rely on local variables and procedure parameters to share data.
- Use control structures like loops and conditional statements to control the flow of execution.
- Document your procedures to make it easier for others to understand and maintain your code.


\subsection{The elements of the RUP methodology, phases and activities}

The Rational Unified Process (RUP) is an iterative and incremental software development methodology that provides a structured approach to managing projects. RUP is organized into four phases, each consisting of several iterations. The methodology is driven by use cases and focuses on managing risk, ensuring quality, and addressing the most critical aspects of the project early on.

RUP is built on six core elements: Roles, Activities, Work Products, Artifacts, Disciplines, and Workflows.

**Phases of RUP:**

1. *Inception*: The main goal of this phase is to establish the project's scope, vision, and business case. Key activities include defining the initial set of requirements, identifying stakeholders, and assessing project feasibility and risks. The inception phase ends with the Lifecycle Objective Milestone, which determines if the project should proceed to the next phase.

2. *Elaboration*: In this phase, the system's architecture is defined, and high-level design decisions are made. The focus is on mitigating risks, refining requirements, and developing a detailed project plan. The elaboration phase ends with the Lifecycle Architecture Milestone, which serves as a checkpoint for architectural stability and overall project health.

3. *Construction*: This phase is where the system is incrementally developed, integrated, and tested. The development team works on implementing use cases, building components, and iteratively testing and refining the system. The construction phase ends with the Initial Operational Capability Milestone, indicating that the system is ready for deployment.

4. *Transition*: In the transition phase, the system is deployed, and end-users are trained and supported. The focus is on fine-tuning the system, addressing any remaining issues, and ensuring a smooth handover to the production environment. The phase ends with the Product Release Milestone, marking the project's completion.

**Activities in RUP:**

Activities in RUP are organized into nine disciplines, which are further divided into specific tasks. The disciplines are:

1. *Business Modeling*: This discipline focuses on understanding the organization's business processes and objectives, helping to align the software solution with business needs.

2. *Requirements*: In this discipline, functional and non-functional requirements are gathered, analyzed, and documented. Use cases are developed to describe the system's behavior from the user's perspective.

3. *Analysis and Design*: This discipline involves designing the system's architecture and components, based on the documented requirements. Design models, class diagrams, and sequence diagrams are created to represent the system's structure and behavior.

4. *Implementation*: In this discipline, the system's components are developed, tested, and integrated. Developers write code, create unit tests, and perform code reviews to ensure quality.

5. *Test*: This discipline focuses on validating that the system meets the specified requirements and is free of defects. Test plans, test cases, and test scripts are created, and various testing techniques (unit, integration, system, and acceptance testing) are employed.

6. *Deployment*: In this discipline, the system is deployed to the production environment and made available to end-users. Deployment plans, installation guides, and user manuals are created to support the deployment process.

7. *Configuration and Change Management*: This discipline involves managing and tracking changes to the system, including requirements, design, code, and documentation. It ensures consistency, traceability, and control throughout the project.

8. *Project Management*: This discipline focuses on planning, organizing, and controlling the project's resources, schedule, and budget. It involves risk management, quality assurance, and progress monitoring.

9. *Environment*: This discipline deals with creating and maintaining the tools, processes, and infrastructure required to support the development team, such as version control systems, build systems, and testing environments.

By following the RUP methodology, organizations can better manage complex software projects, minimize risks, and improve the quality of their software solutions.


\section{Set theory basics}

Sets, operations on sets. Relations, functions. Injective, onto, and bijective functions. Equivalence relations. Ordering relations. Natural, integer, rational, real, and complex numbers: their properties, operations and ordering properties.


\section{Logics}

Propositional logics: propositions (statements), operations with propositions, formulas, formalization, disjunctive and conjunctive normal forms, inferences, inference rules. Predicate logic: predicates, quantifiers, formulae, formalization and interpretation, inferences in predicate logics.


\section{Linear algebra}

Real vector spaces, normed spaces. Vector operations. Inner (scalar) product. Linear independence, basis, dimension. Matrices and linear operators. Homogeneous and inhomogeneous linear systems of equations. The determinant and trace of matrices. Eigenvalues and eigenvectors, spectral decomposition of symmetric matrices.


\section{Calculus}

Convergence of sequences and series. Taylor-series. Limits and continuity of functions. Differential calculus of one- and multivariable functions. Finding function minima and maxima. Convexity of functions. Real integral calculus, definite and indefinite integrals.


\section{Probability theory}

Discrete and continuous random variables. Probability distributions and probability densities. Independence of random variables. Joint probabilities, marginal distributions, expectation value, variance, correlation. Laws of large numbers. Central limit theorem.


\section{Number theory}

Greatest common divisor. Euclidean division. Euclidean algorithm. Primes and non-factorizable numbers. Unique prime factorization. Systems of linear Diophantine equations, linear congruences. Euler’s theorem, Chinese remainder theorem. Number theoretic functions. Multiplicativity. Sum and inverse functions.


\end{document}
