\documentclass{article}

\usepackage[a4paper, margin=2cm]{geometry}


\begin{document}

\section{System engineering and software development}

Basics of UML. Meta-models. The 4-layer meta-model of UML. Basics of workflow modeling. The description of the steps of the classic and RUP methodologies with the Software Process Engineering Model (SPEM) UML profile. Programming and software engineering technologies and methodologies. Procedural techniques. The elements of the RUP methodology, phases and activities.


\subsection{Basics of UML}

Unified Modeling Language (UML) is a standardized visual language used to model software systems, including their structure, behavior, and interactions. UML provides a set of graphical notation techniques to create abstract models of specific systems, which can then be used as blueprints for actual software development.

Here are the basic building blocks of UML:

1. **UML Diagrams**: UML is comprised of various types of diagrams, each serving a specific purpose. The main diagrams are:

   - Class Diagram
   - Object Diagram
   - Use Case Diagram
   - Sequence Diagram
   - Collaboration Diagram
   - Statechart Diagram
   - Activity Diagram
   - Component Diagram
   - Deployment Diagram

2. **UML Elements**: Each UML diagram consists of several elements, which represent the various aspects of a software system. Some common UML elements are:

   - Classes and Objects
   - Interfaces
   - Relationships (association, aggregation, composition, inheritance, and dependency)
   - Packages
   - Actors
   - Use Cases
   - States and Transitions
   - Activities and Actions
   - Components and Nodes

Let's briefly explain some of the most important UML diagrams:

- **Class Diagram**: This diagram shows the static structure of a system by depicting its classes, attributes, operations, and relationships between objects. It represents the fundamental blueprint of a software system.

- **Object Diagram**: This diagram represents the instances of classes and their relationships at a particular point in time. It is used to visualize the structure of objects and their relationships in the system.

- **Use Case Diagram**: This diagram models the interactions between actors (end-users or other systems) and the system itself. It helps identify the system's functionality and the roles that interact with it.

- **Sequence Diagram**: This diagram represents the dynamic behavior of a system by showing the interactions between objects in a specific sequence. It highlights the order in which messages are exchanged between objects to accomplish a specific task.

- **Statechart Diagram**: This diagram models the dynamic behavior of an object by showing its possible states and the transitions between them. It is useful for representing the life cycle of an object over time.

- **Activity Diagram**: This diagram models the flow of control within a system, showing the sequence of activities and decisions involved in the execution of a specific process or use case.

- **Component Diagram**: This diagram represents the physical organization of a software system, displaying the components that make up the system and their dependencies.

- **Deployment Diagram**: This diagram models the hardware architecture of a system, depicting the nodes (hardware devices) and the components that are deployed on them.

These are just the basics of UML. To fully understand and utilize UML, it is essential to learn the various diagrams and elements in detail and practice creating models for real-world software systems.


\subsection{Meta-models. The 4-layer meta-model of UML}

A meta-model is a model that describes the structure, semantics, and constraints of other models. In other words, it is a model of models. Meta-models are used to define the rules and conventions of modeling languages, ensuring that the models created are consistent and adhere to a set of predefined guidelines.

In the context of UML (Unified Modeling Language), the UML meta-model is a specification that defines the structure and semantics of UML diagrams and their elements. The UML meta-model acts as a blueprint for creating UML models that are both consistent and compliant with the UML standards.

There are typically four layers in the meta-modeling hierarchy:

1. **M0 - Instance Layer**: This layer represents the actual instances or objects in the real-world system being modeled.
2. **M1 - Model Layer**: This layer contains the models that describe the structure and behavior of the system. These models are created using the modeling language defined by the meta-model in the M2 layer.
3. **M2 - Meta-Model Layer**: This layer contains the meta-model that defines the structure, semantics, and constraints of the modeling language used in the M1 layer.
4. **M3 - Meta-Meta-Model Layer**: This layer contains the meta-meta-model that describes the structure and semantics of the meta-models in the M2 layer. The most widely known meta-meta-model is the MOF (Meta-Object Facility) standard, which is used to define the structure of various meta-models, including the UML meta-model.

In summary, meta-models are essential for establishing the structure and rules of modeling languages. They provide the foundation for creating consistent and compliant models that accurately represent the systems being modeled. Meta-modeling is a crucial aspect of model-driven engineering and software development methodologies.


\subsection{Basics of workflow modeling}
\subsection{Basics of workflow modeling.}

Workflow modeling is a technique used to represent the sequence of activities, tasks, and decisions involved in a business process or system. The goal of workflow modeling is to analyze and optimize these processes, improve efficiency, and facilitate communication among team members.

Here are the basics of workflow modeling:

1. **Identify the Process**: Determine the process you want to model. It could be a high-level business process or a more specific, low-level process within a larger system. Make sure you clearly define the scope and purpose of the process.

2. **Define Activities**: Break down the process into individual activities or tasks. Activities represent the work performed in the process, such as creating a document, approving a request, or updating a database.

3. **Determine Sequence**: Establish the order in which activities must be executed. Some activities may need to be completed before others can begin, while others may be performed concurrently.

4. **Identify Roles**: Assign roles (or actors) to each activity. Roles can represent people, teams, or systems responsible for performing the activities. Clearly defining roles helps to establish accountability and improve communication within the process.

5. **Model Decisions and Branching**: Identify decision points in the process where the workflow may follow different paths based on certain conditions. These decision points may involve branching (splitting the flow into multiple paths) or merging (combining multiple paths back into a single path).

6. **Add Workflow Elements**: In addition to activities and decision points, there are several other elements commonly used in workflow modeling, such as:

   - Start and end events: Indicate the beginning and end of the process.
   - Gateways: Represent decision points or points where the flow splits or merges.
   - Connectors: Show the flow between activities, events, and gateways.

7. **Choose a Notation**: Select a notation or modeling language to visually represent your workflow model. Some popular notations include Business Process Model and Notation (BPMN), UML Activity Diagrams, and flowcharts. Each notation has its own set of symbols and conventions for representing workflows.

8. **Create the Workflow Diagram**: Using the chosen notation, create a visual representation of the workflow by arranging the elements on a canvas or diagramming tool. Make sure the diagram is clear, easy to understand, and accurately represents the process being modeled.

9. **Validate and Refine**: Review the workflow model with stakeholders and experts to ensure that it accurately represents the process and meets the intended objectives. Refine the model as needed based on feedback and analysis.

10. **Monitor and Improve**: Once the workflow model is implemented, monitor its performance and gather data on areas that may need improvement. Continuously refine the model to optimize the process and adapt to changing requirements.

By following these basic steps, you can create a workflow model that represents and optimizes business processes, leading to improved efficiency, better communication, and more effective decision-making.



\subsection{The description of the steps of the classic and RUP methodologies with the Software Process Engineering Model (SPEM) UML profile}

The Software Process Engineering Meta-model (SPEM) is a UML profile that provides a standardized way to model, define, and manage software development processes. It allows organizations to customize and adapt methodologies like the classic Waterfall model and Rational Unified Process (RUP) to their specific needs.

Let's describe the steps of the classic Waterfall model and RUP using SPEM concepts:

**Classic Waterfall Model**

The Waterfall model is a linear and sequential approach to software development. It consists of distinct phases, with each phase being completed before moving on to the next.

1. *Requirements*: In this phase, the project's functional and non-functional requirements are gathered and documented. In SPEM, this phase can be represented using a **Task Definition** called "Gather Requirements," assigned to a **Role Definition** such as "Business Analyst."

2. *Design*: The system architecture and design are created based on the requirements. In SPEM, this phase can be represented by a Task Definition called "System Design," assigned to a Role Definition such as "System Architect."

3. *Implementation*: The design is translated into source code by developers. In SPEM, this phase can be represented by a Task Definition called "Implement System," assigned to a Role Definition such as "Software Developer."

4. *Testing*: The system is tested to ensure it meets the requirements and is free of defects. In SPEM, this phase can be represented by a Task Definition called "Test System," assigned to a Role Definition such as "Test Engineer."

5. *Deployment*: The system is deployed to the production environment and made available to end-users. In SPEM, this phase can be represented by a Task Definition called "Deploy System," assigned to a Role Definition such as "Deployment Engineer."

6. *Maintenance*: The system is maintained and updated as needed. In SPEM, this phase can be represented by a Task Definition called "Maintain System," assigned to a Role Definition such as "Maintenance Engineer."

**Rational Unified Process (RUP)**

RUP is an iterative and incremental software development process, organized into four phases, each consisting of several iterations.

1. *Inception*: The project's scope, vision, and business case are established. In SPEM, this phase can include Task Definitions such as "Define Project Vision" and "Create Business Case," assigned to Role Definitions like "Project Manager" and "Business Analyst."

2. *Elaboration*: The system's architecture and high-level design are defined, and risks are identified and mitigated. In SPEM, this phase can include Task Definitions like "Define System Architecture," "Identify Risks," and "Develop Risk Mitigation Plan," assigned to Role Definitions such as "System Architect" and "Risk Manager."

3. *Construction*: The system is incrementally developed, integrated, and tested. In SPEM, this phase can include Task Definitions like "Develop Features," "Integrate Components," and "Test Increment," assigned to Role Definitions like "Software Developer" and "Test Engineer."

4. *Transition*: The system is deployed, and end-users are trained and supported. In SPEM, this phase can include Task Definitions like "Deploy System," "Train Users," and "Provide Support," assigned to Role Definitions like "Deployment Engineer" and "Training Specialist."

In both methodologies, SPEM **Work Product Definitions** can be used to represent the artifacts produced during the process, such as requirements documents, design documents, source code, test plans, and user manuals. 

By using the SPEM UML profile, organizations can model their software development processes and customize methodologies like the Waterfall model and RUP to better fit their specific needs and context.


\subsection{Programming and software engineering technologies and methodologies}

Programming and software engineering technologies and methodologies are the tools, techniques, and approaches used to develop, manage, and maintain software systems. They are designed to improve the quality, efficiency, and effectiveness of the software development process. Here's a list of some commonly used technologies and methodologies in software engineering:

**Programming Languages**

These are the languages used to write software programs. Some popular programming languages include:

1. Python
2. Java
3. JavaScript
4. C++
5. C#
6. Ruby
7. PHP
8. Swift
9. Kotlin
10. TypeScript

**Integrated Development Environments (IDEs)**

These are software applications that provide a comprehensive set of tools for software development, such as code editing, debugging, and version control integration. Some popular IDEs are:

1. Visual Studio Code
2. IntelliJ IDEA
3. Eclipse
4. Xcode
5. PyCharm
6. Android Studio
7. NetBeans
8. Sublime Text
9. Atom

**Version Control Systems**

These systems help manage and track changes to software code, allowing developers to collaborate effectively and maintain a history of modifications. Some popular version control systems are:

1. Git
2. Subversion (SVN)
3. Mercurial
4. Perforce
5. Concurrent Versions System (CVS)

**Software Development Methodologies**

These are structured approaches to software development that help manage and organize the process. Some popular methodologies include:

1. Agile (e.g., Scrum, Kanban, Extreme Programming)
2. Waterfall
3. Rational Unified Process (RUP)
4. Spiral
5. V-Model
6. DevOps
7. Lean Software Development

**Project Management and Collaboration Tools**

These tools help plan, coordinate, and track software development projects, facilitating communication among team members. Some popular tools are:

1. Jira
2. Trello
3. Asana
4. Microsoft Project
5. Basecamp
6. Monday.com
7. ClickUp
8. GitHub Projects

**Automated Testing and Continuous Integration Tools**

These tools help automate testing and integration processes, improving the quality and efficiency of software development. Some popular tools include:

1. Jenkins
2. Travis CI
3. TeamCity
4. CircleCI
5. GitLab CI/CD
6. Selenium
7. JUnit
8. TestNG

**Software Design Patterns**

These are reusable solutions to common problems encountered during software design. Some popular design patterns include:

1. Singleton
2. Factory Method
3. Abstract Factory
4. Observer
5. Strategy
6. Decorator
7. Adapter
8. Facade

**Software Architectures**

These are high-level structures and patterns used to organize software systems. Some popular architectures include:

1. Monolithic
2. Microservices
3. Service-Oriented Architecture (SOA)
4. Layered Architecture
5. Event-Driven Architecture
6. Model-View-Controller (MVC)
7. Model-View-ViewModel (MVVM)

These technologies and methodologies are just a starting point. As a software engineer, it's essential to stay up-to-date with the latest trends and best practices in the industry to continue improving and delivering high-quality software systems.


\subsection{Procedural techniques}

Procedural techniques are a programming paradigm that focuses on using procedures (also known as routines, subroutines, or functions) to solve problems. The main idea behind procedural programming is to break down a complex problem into smaller, more manageable tasks. These tasks are then implemented as procedures, which can be called and reused throughout the program. Procedural programming is typically contrasted with object-oriented programming, which focuses on organizing code around objects and their interactions.

Here are some key features of procedural programming:

1. **Sequential Execution**: Procedural programming follows a top-down approach, with code executed line by line in a sequential manner.

2. **Procedures**: The code is organized into reusable procedures, which can be called by other parts of the program. Procedures are typically defined with a specific input (parameters) and output (return value), and they may have local variables.

3. **Modular Design**: Procedural programming promotes modular design, where each procedure is responsible for a specific task. This approach makes it easier to understand, maintain, and modify the code.

4. **Control Structures**: Procedural programming languages provide control structures such as loops (for, while) and conditional statements (if-else) to control the flow of execution.

5. **Global and Local Variables**: Variables in procedural programming can be global (accessible throughout the entire program) or local (accessible only within the procedure in which they are defined). Local variables help to limit the scope of variables and reduce the risk of unintended side effects.

Some popular procedural programming languages include:

1. C
2. Pascal
3. Fortran
4. COBOL
5. BASIC

Many modern programming languages, such as Python, JavaScript, and Java, also support procedural programming techniques alongside other paradigms like object-oriented and functional programming. This allows programmers to choose the best approach for a given problem or combine paradigms as needed.

To use procedural techniques effectively, keep the following principles in mind:

- Break down complex problems into smaller, more manageable tasks.
- Encapsulate tasks within well-defined procedures with clear inputs and outputs.
- Minimize the use of global variables and rely on local variables and procedure parameters to share data.
- Use control structures like loops and conditional statements to control the flow of execution.
- Document your procedures to make it easier for others to understand and maintain your code.


\subsection{The elements of the RUP methodology, phases and activities}

The Rational Unified Process (RUP) is an iterative and incremental software development methodology that provides a structured approach to managing projects. RUP is organized into four phases, each consisting of several iterations. The methodology is driven by use cases and focuses on managing risk, ensuring quality, and addressing the most critical aspects of the project early on.

RUP is built on six core elements: Roles, Activities, Work Products, Artifacts, Disciplines, and Workflows.

**Phases of RUP:**

1. *Inception*: The main goal of this phase is to establish the project's scope, vision, and business case. Key activities include defining the initial set of requirements, identifying stakeholders, and assessing project feasibility and risks. The inception phase ends with the Lifecycle Objective Milestone, which determines if the project should proceed to the next phase.

2. *Elaboration*: In this phase, the system's architecture is defined, and high-level design decisions are made. The focus is on mitigating risks, refining requirements, and developing a detailed project plan. The elaboration phase ends with the Lifecycle Architecture Milestone, which serves as a checkpoint for architectural stability and overall project health.

3. *Construction*: This phase is where the system is incrementally developed, integrated, and tested. The development team works on implementing use cases, building components, and iteratively testing and refining the system. The construction phase ends with the Initial Operational Capability Milestone, indicating that the system is ready for deployment.

4. *Transition*: In the transition phase, the system is deployed, and end-users are trained and supported. The focus is on fine-tuning the system, addressing any remaining issues, and ensuring a smooth handover to the production environment. The phase ends with the Product Release Milestone, marking the project's completion.

**Activities in RUP:**

Activities in RUP are organized into nine disciplines, which are further divided into specific tasks. The disciplines are:

1. *Business Modeling*: This discipline focuses on understanding the organization's business processes and objectives, helping to align the software solution with business needs.

2. *Requirements*: In this discipline, functional and non-functional requirements are gathered, analyzed, and documented. Use cases are developed to describe the system's behavior from the user's perspective.

3. *Analysis and Design*: This discipline involves designing the system's architecture and components, based on the documented requirements. Design models, class diagrams, and sequence diagrams are created to represent the system's structure and behavior.

4. *Implementation*: In this discipline, the system's components are developed, tested, and integrated. Developers write code, create unit tests, and perform code reviews to ensure quality.

5. *Test*: This discipline focuses on validating that the system meets the specified requirements and is free of defects. Test plans, test cases, and test scripts are created, and various testing techniques (unit, integration, system, and acceptance testing) are employed.

6. *Deployment*: In this discipline, the system is deployed to the production environment and made available to end-users. Deployment plans, installation guides, and user manuals are created to support the deployment process.

7. *Configuration and Change Management*: This discipline involves managing and tracking changes to the system, including requirements, design, code, and documentation. It ensures consistency, traceability, and control throughout the project.

8. *Project Management*: This discipline focuses on planning, organizing, and controlling the project's resources, schedule, and budget. It involves risk management, quality assurance, and progress monitoring.

9. *Environment*: This discipline deals with creating and maintaining the tools, processes, and infrastructure required to support the development team, such as version control systems, build systems, and testing environments.

By following the RUP methodology, organizations can better manage complex software projects, minimize risks, and improve the quality of their software solutions.


\section{Set theory basics}

Sets, operations on sets. Relations, functions. Injective, onto, and bijective functions. Equivalence relations. Ordering relations. Natural, integer, rational, real, and complex numbers: their properties, operations and ordering properties.


\section{Logics}

Propositional logics: propositions (statements), operations with propositions, formulas, formalization, disjunctive and conjunctive normal forms, inferences, inference rules. Predicate logic: predicates, quantifiers, formulae, formalization and interpretation, inferences in predicate logics.


\section{Linear algebra}

Real vector spaces, normed spaces. Vector operations. Inner (scalar) product. Linear independence, basis, dimension. Matrices and linear operators. Homogeneous and inhomogeneous linear systems of equations. The determinant and trace of matrices. Eigenvalues and eigenvectors, spectral decomposition of symmetric matrices.


\section{Calculus}

Convergence of sequences and series. Taylor-series. Limits and continuity of functions. Differential calculus of one- and multivariable functions. Finding function minima and maxima. Convexity of functions. Real integral calculus, definite and indefinite integrals.


\section{Probability theory}

Discrete and continuous random variables. Probability distributions and probability densities. Independence of random variables. Joint probabilities, marginal distributions, expectation value, variance, correlation. Laws of large numbers. Central limit theorem.


\section{Number theory}

Greatest common divisor. Euclidean division. Euclidean algorithm. Primes and non-factorizable numbers. Unique prime factorization. Systems of linear Diophantine equations, linear congruences. Euler’s theorem, Chinese remainder theorem. Number theoretic functions. Multiplicativity. Sum and inverse functions.


\end{document}
