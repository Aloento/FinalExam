\documentclass[11pt,journal,compsoc]{IEEEtran}

\usepackage{ctex}

\usepackage{graphicx}

\usepackage{url}

\usepackage{hyperref}

\usepackage{enumitem}

\usepackage{amssymb}

\usepackage{indentfirst}

\setlength{\parindent}{0em}

\begin{document}


Informatics


\section{Processors \& Memories}


\subsection{Von Neumann Principle}

程序 和 数据 储存在 单一独立的储存单元 内存 中,它们通过 共同的总线 进行访问。它们在 物理上 是相同的,计算机根据 指令 来处理它们,易重新编程。


\subsection{Von Neumann Loop}

\begin{enumerate}
    \item CPU 从内存中取指令 Fetch
    \item 解码以确定操作 Decode
    \item 执行 Execute
    \item 从内存中取
    \item 写回内存
\end{enumerate}

重复,直到程序结束。CPU 在获取 指令和数据 之间交替,降低速度。


\subsection{Von Neumann Architecture}

存储器、运算器、控制器、输入设备和输出设备五部分组成计算机。运算器和控制器合称为中央处理器。


\subsection{Harvard}

此架构使用 两个不同的储存器 来 分别储存 指令和数据,不同的总线进行访问,速度更快,成本更高。


\subsection{CISC}

Complex / Reduced Instruction Set Computer

复杂指令集的每个指令可执行若干低端操作,指令数目多而复杂,周期长,更适合复杂操作,简单操作可能浪费时间。

原因:寄存器昂贵,尽可能使单个指令做更多的工作


\subsection{RISC}

精简指令集针对流水线化的处理器优化,指令数目少,周期短,更好的并行执行,编译器的效率更高,便宜节能。

x86 对简单指令有硬件加速。

ARM V7 和之前没有 AES。


\subsection{Commands}

一组指示处理器执行特定操作的二进制代码,包含 OpCode 和 Operand 两部分,汇编。


\subsection{Registers}

CPU 内部用来存放临时数据的内存,非常快,很小。x86-64架构有 16 个通用寄存器。

CPU $\leftrightarrow$ 寄存器 $\leftrightarrow$ 缓存 $\leftrightarrow$ 内存


\subsection{Pipeline}

指令流水线,将每条指令分解为多步,每个阶段可以并行执行不同的指令,从而实现几条指令并行处理。指令仍是一条条顺序执行,一个时钟周期内可以执行一条完整的指令。当一个指令在 Execute 时,另一个指令可以开始执行 Decode,以此类推。


\subsection{Superscalar}

在单个核心中有多个流水线,一个时钟周期内执行多个指令,并行地对多条指令进行流水处理。


\subsection{Memories}

\paragraph{按物理结构}

\begin{description}
    \item[SRAM] Field-Effect Transistor、无需刷新、极快、\\ 昂贵、CPU 缓存

    \item[DRAM] Capacitance、周期刷新、内存

    \item[ROM] 只读

    \item[Programmable ROM] 内部有熔丝,利用电流将其烧断,以写入数据,一经烧断便无法恢复

    \item[Erasable PROM] 利用高电压将数据写入,有透明窗口,曝光于紫外线时抹除数据

    \item[OneTime PROM] 与 EPROM 一样,但无透明窗

    \item[Electrically EPROM] 使用高电场来抹除

    \item[Flash] NAND($!\&$ SSD)与 NOR($!\|$ BIOS)型、\\ 寿命有限、较慢
\end{description}

\paragraph{按组织方式}

\begin{description}
    \item[Serial] 比如 Magnetic Tape、密度高、速度慢、\\ 不能随机存取

    \item[Parallel] 比如 Disk、同时读写多个比特

    \item[Distributed] 信息存储于多个独立且互不干扰的 \\ 设备中、可靠性、扩展性

    \item[Hierarchical] 高效率,比如 Cache -> RAM -> ROM
\end{description}


\paragraph{按访问类型}

\begin{description}
    \item[Direct Memory Access] 不经过 CPU,比如显卡的 DirectStorage

    \item[Sequential Access Memory] 磁带

    \item[Random Access Memory] 内存
\end{description}


\section{Peripherals \& Storage \& I/O}


\subsection{Storage}

\subsubsection{By Construction}

\begin{description}
    \item[Solid] SSD, Flash 速度快

    \item[Optical] CD, use Laser

    \item[Magnetic] HDD, Tape, Floppy 容量大
\end{description}

\subsubsection{By Operation Principle}

\begin{description}
    \item[RAM] Volatile
    \item[ROM] Non-Volatile
    \item[Sequential]
\end{description}


\subsection{Peripherals}

\subsubsection{By Functionality}

\begin{description}
    \item[Input] Keyboard, Mouse
    
    \item[Output] Monitor, Speaker

    \item[Storage] Flash Drive

    \item[Network] Router, Switch
\end{description}

\subsubsection{By Means}

\begin{description}
    \item[Wired]
    \item[Wireless]
\end{description}

还有 Bidirectional 双向设备


\subsection{Bus}

is physical pathways used by CPU and other devices to communicate.


\subsubsection{Data}

CPU \leftrightarrow RAM

CPU \leftrightarrow I/O Devices

传输数据,宽度决定一次传输多少比特,8 - 64 bits 常见


\subsubsection{Address}

沟通物理地址,宽度决定寻址的内存大小


\subsubsection{Control}

控制 CPU 和设备的:读写、时钟、中断、复位 等信号


\subsection{I/O}

handling is the process of managing the communication between CPU and devices.


\subsubsection{Programmed}

最简单,基本,低效率。CPU 直接控制 I/O 设备和 存储器 之间的数据传输


\subsubsection{IRQ}

Interrupt-Request。I/O 需要运算时,向 CPU 发送中断请求,并处理需求


\subsubsection{Direct Memory Access}


\subsection{Protocols}

传输数据的方法


\subsubsection{Parallel}

所有信号线在同一时刻传输数据 导致信号互相干扰(串扰)。传输速率提高,时钟同步和时序控制变难。

\begin{description}
    \item[Centronics] Enhanced Parallel Port 打印机通信,8 位数据总线和几个控制信号,25 PIN

    \item[Peripheral Component Interconnect] 连接外围设备,32 位 133 MB/s
\end{description}


\subsubsection{Serial}

一次传输一比特,数据沿单一信道传输,差分信号降低干扰和提高质量,增加 Lane 数量提升速率。

差分:一正一负两根线,通过电压差获取数据。

\begin{description}
    \item[Recommended Standard 232] 一条数据线发送接收,控制线用于握手和流量控制,工业

    \item[High-Definition Multimedia Interface] DP

    \item[PCI Express] x32 表示 Lane 数,32 GT/s 每 Lane
\end{description}


\subsubsection{Universal Serial Bus}

同时供电通信,即插即用,40 Gbps,Thunderbolt 兼容

\begin{description}
    \item[主从通信] Host 管理控制数据传输

    \item[层次结构] ~
    
        \begin{itemize}
            \item 物理:接口形状,功率等
            \item 链路:建立和维护通信
            \item 协议:格式等
            \item 传输:打包数据
            \item 会话:管理
            \item 应用:处理驱动程序的请求
        \end{itemize}

    \item[数据包] 传输的基本单位,Token、Data 和 Handshake

    \item[传输模式] ~

        \begin{itemize}
            \item Control:设备配置,状态查询,命令执行

            \item Bulk:大量数据,高速,延迟高

            \item Interrupt:低延迟,少量数据,实时,慢,鼠标

            \item Isochronous:音视频,固定速率和延迟,会丢包
        \end{itemize}

    \item[初始化] 识别设备,查询信息,分配地址,选择配置
\end{description}


\section{Operating Systems}




\section{Network Architectures}


\section{Network \& Security}


\section{Databases}


\section{Algorithms \& Data Structures}


\section{Graph Algorithms}


\section{Programming}


\section{Compilers}


\section{System Engineering}


Mathematics


\section{Set}


\section{Logics}


\section{Linear Algebra}


\section{Calculus}


\section{Probability}


\section{Numerical}


\end{document}
