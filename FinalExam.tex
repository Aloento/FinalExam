\documentclass[11pt,journal,compsoc]{IEEEtran}

\usepackage{ctex}

\usepackage{graphicx}

\usepackage{url}

\usepackage{hyperref}

\usepackage{enumitem}

\usepackage{amssymb}

\usepackage{indentfirst}

\setlength{\parindent}{0em}

\begin{document}


\section{Informatics}


\subsection{Processors \& Memories}


\subsubsection{Von Neumann Principle}

程序 和 数据 储存在 单一独立的储存单元 内存 中,它们通过 共同的总线 进行访问。它们在 物理上 是相同的,计算机根据 指令 来处理它们,更容易重新编程。


\subsubsection{Von Neumann Loop}

\begin{enumerate}
    \item CPU 从内存中取指令 Fetch
    \item 解码以确定操作 Decode
    \item 执行 Execute
    \item 从内存中取
    \item 写回内存
\end{enumerate}

重复,直到程序结束。CPU 在获取 指令和数据 之间交替,降低速度。


\subsubsection{Von Neumann Architecture}

存储器、运算器、控制器、输入设备和输出设备五部分组成计算机。运算器和控制器合称为中央处理器。


\subsubsection{Harvard}

此架构使用 两个不同的储存器 来 分别储存 指令和数据,不同的总线进行访问,速度更快,成本更高。


\subsubsection{CISC}

Complex / Reduced Instruction Set Computer

复杂指令集的每个指令可执行若干低端操作,指令数目多而复杂,周期长,更适合复杂操作,简单操作可能浪费时间。

寄存器昂贵,尽可能使单个指令做更多的工作


\subsubsection{RISC}

精简指令集针对流水线化的处理器优化,指令数目少,周期短,更好的并行执行,编译器的效率更高,便宜节能。

x86 对简单指令有硬件加速。

ARM V7 和之前没有 AES。


\subsubsection{Commands}

一组指示处理器执行特定操作的二进制代码,包含 OpCode 和 Operand 两部分,汇编。


\subsubsection{Registers}

CPU 内部用来存放临时数据的内存,非常快,很小。x86-64架构有 16 个通用寄存器。

CPU $\leftrightarrow$ 寄存器 $\leftrightarrow$ 缓存 $\leftrightarrow$ 内存


\subsubsection{Pipeline}

指令流水线,将每条指令分解为多步,每个阶段可以并行执行不同的指令,从而实现几条指令并行处理。指令仍是一条条顺序执行,一个时钟周期内可以执行一条完整的指令。当一个指令在 Execute 时,另一个指令可以开始执行 Decode,以此类推。


\subsubsection{Superscalar}

在单个核心中有多个流水线,一个时钟周期内执行多个指令,并行地对多条指令进行流水处理。


\subsubsection{Memories}

按物理结构:

\begin{description}
    \item[SRAM] Field-Effect Transistor、无需刷新、极快、昂贵、CPU 缓存

    \item[DRAM] Capacitance、周期刷新、内存

    \item[ROM] 只读

    \item[Programmable ROM] 内部有熔丝,利用电流将其烧断,以写入数据,一经烧断便无法恢复

    \item[Erasable PROM] 利用高电压将数据写入,有透明窗口,曝光于紫外线时抹除数据

    \item[OneTime PROM] 与 EPROM 一样,但不设置透明窗

    \item[Electrically EPROM] 使用高电场来抹除

    \item[Flash] NAND(SSD)与 NOR(BIOS)型、寿命有限、较慢
\end{description}


按组织方式:

\begin{description}
    \item[Serial] 比如 Magnetic Tape、密度高、速度慢、不能随机存取

    \item[Parallel] 比如 Disk、同时读写多个比特

    \item[Distributed] 信息存储于多个独立且互不干扰的设备中、可靠性、扩展性

    \item[Hierarchical] 高效率,比如 Cache -> RAM -> ROM
\end{description}


按访问类型:

\begin{description}
    \item[Direct Memory Access] 不经过 CPU,比如显卡的 DirectStorage

    \item[Sequential Access Memory] 磁带

    \item[Random Access Memory] 内存
\end{description}


\subsection{Peripherals \& Storage \& I/O}


\subsubsection{Storage}

By Construction:

\begin{description}
    \item[Solid] SSD, Flash 速度快

    \item[Optical] CD, use Laser

    \item[Magnetic] HDD, Tape, Floppy 容量大
\end{description}

By Operation Principle:

\begin{description}
    \item[RAM] Volatile
    \item[ROM] Non-Volatile
    \item[Sequential]
\end{description}


\subsubsection{Peripherals}

By Functionality:

\begin{description}
    \item[Input] Keyboard, Mouse
    
    \item[Output] Monitor, Speaker

    \item[Storage] Flash Drive

    \item[Network] Router, Switch
\end{description}

By Means:

\begin{description}
    \item[Wired]
    \item[Wireless]
\end{description}

还有 Bidirectional 双向设备


\subsection{Bus}

is physical pathways used by CPU and other devices to communicate.


\subsubsection{Data}

CPU \leftrightarrow RAM

CPU \leftrightarrow I/O Devices

传输数据,宽度决定一次传输多少比特,常见 8 - 64 bits


\subsubsection{Address}

沟通物理地址,宽度决定寻址的内存大小


\subsubsection{Control}

控制 CPU 和设备的:读写、时钟、中断、复位 等信号


\subsection{I/O}

handling is the process of managing the communication between CPU and devices.



\subsection{Operating Systems}


\subsection{Network Architectures}


\subsection{Network \& Security}


\subsection{Databases}


\subsection{Algorithms \& Data Structures}


\subsection{Graph Algorithms}


\subsection{Programming}


\subsection{Compilers}


\subsection{System Engineering}


\section{Mathematics}


\subsection{Set}


\subsection{Logics}


\subsection{Linear Algebra}


\subsection{Calculus}


\subsection{Probability}


\subsection{Numerical}


\end{document}
